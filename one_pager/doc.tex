\documentclass{article}

% use Times
\usepackage{times}
% For figures
\usepackage{graphicx} % more modern
%\usepackage{epsfig} % less modern
\usepackage{subfigure}
% For color
\usepackage{color}
\usepackage[utf8]{inputenc}  

% For citations
\usepackage{natbib}

% For algorithms
\usepackage{algorithm}
\usepackage{algorithmic}

%\usepackage{fullpage}
\usepackage{hyperref}
\usepackage{authblk}

% Packages from the older tex file
\usepackage{amsmath}
\usepackage{amsthm}
\usepackage{amssymb}
\usepackage{graphicx}
\usepackage{bbm}
%%%%%%%%%%%%%%%%%%%%%%%%%%%%%%
%%% Header for thm and def %%%
%%%%%%%%%%%%%%%%%%%%%%%%%%%%%%
% Theorem Styles
\newtheorem{theorem}{Theorem}[section]
\newtheorem{lemma}[theorem]{Lemma}
\newtheorem{proposition}[theorem]{Proposition}
\newtheorem{corollary}[theorem]{Corollary}
% Definition Styles
\newtheorem{definition}{Definition}[section]
\newtheorem{example}{Example}[section]
\newtheorem{remark}{Remark}
%\renewcommand{\qedsymbol}{$\blacksquare$}
%%%%%%%%%%%%%%%%%%%%
%%% New commands %%%
%%%%%%%%%%%%%%%%%%%%
\newcommand\numberthis{\addtocounter{equation}{1}\tag{\theequation}}
%%%%%%%%%%%%%%%%%%%%%%%%%%%%%%%
\usepackage{hyperref}
% Packages hyperref and algorithmic misbehave sometimes.  We can fix
% this with the following command.
\newcommand{\theHalgorithm}{\arabic{algorithm}}
\newcommand{\grad}{\nabla}
\newcommand{\E}{\mathbb E}
\newcommand{\R}{\mathbb R}
\renewcommand{\P}{\mathbb P}
%%%%%%%%%%%%%%%%%%%%%%%%%%%%%%
%%% Header for thm and def %%%
%%%%%%%%%%%%%%%%%%%%%%%%%%%%%%
% Theorem Styles
\newtheorem{assump}{}
\renewcommand{\theassump}{Assumption~\arabic{assump}}
%\renewcommand{\qedsymbol}{$\blacksquare$}

% Packages hyperref and algorithmic misbehave sometimes.  We can fix
% this with the following command.
\newcommand{\bG}{\boldsymbol{G}}
\newcommand{\be}{\boldsymbol{e}}
\newcommand{\bb}{\textbf{\textsf{b}}}
\newcommand{\bC}{\boldsymbol{C}}
\newcommand{\blam}{\boldsymbol{\lambda}}
\newcommand{\blamt}{\boldsymbol{\lambda_t}}
\newcommand{\bK}{\textbf{\textsf{K}}}
\newcommand{\bI}{\boldsymbol{I_d}}
\newcommand{\bSig}{\boldsymbol{\Sigma}}
\newcommand{\bLam}{\boldsymbol{\Lambda}}
\newcommand{\bL}{\boldsymbol{L}}
\newcommand{\bR}{\boldsymbol{R}}
\newcommand{\bt}{\boldsymbol{t}}
\newcommand{\bM}{\boldsymbol{M}}
\newcommand{\bO}{\boldsymbol{O}}
\newcommand{\bD}{\boldsymbol{D}}
\newcommand{\bx}{\boldsymbol{x}}
\newcommand{\bX}{\boldsymbol{X}}
\newcommand{\bY}{\boldsymbol{Y}}
\newcommand{\bA}{\boldsymbol{A}}
\newcommand{\bQ}{\boldsymbol{Q}}
\newcommand{\bB}{\boldsymbol{B}}
\newcommand{\bN}{\boldsymbol{N}}
\newcommand{\bU}{\boldsymbol{U}}
\newcommand{\bKc}{\boldsymbol{K^c}}
\newcommand{\bmu}{\boldsymbol{\mu}}
\newcommand{\prox}{\mbox{\textbf{prox}}}
%\newcommand{\E}{\mathbb{E}}
%\newcommand{\R}{\mathbb{R}}
\newcommand{\bW}{\boldsymbol{W}}
\newcommand{\bPsi}{\boldsymbol{\Psi}}
\newcommand{\bPhi}{\boldsymbol{\Phi}}
\newcommand{\bdM}{\boldsymbol{dM_t}}

\newcommand{\footremember}[2]{%
    \footnote{#2}
    \newcounter{#1}
    \setcounter{#1}{\value{footnote}}%
}
\newcommand{\footrecall}[1]{%
    \footnotemark[\value{#1}]%
}
\newcommand{\overbar}[1]{\mkern 1.5mu\overline{\mkern-1.5mu#1\mkern-1.5mu}\mkern 1.5mu}


\title{Concours G\'en\'eration Blockchain : One-pager d'introduction}

\author[1]{Massil Achab \& Mastane Achab, Ecole Polytechnique}

\renewcommand\Affilfont{\itshape\small}
\begin{document}

\maketitle

\section*{Energy retail market}

\subsection*{Contexte et identification du besoin}
Le domaine de l'\'energie a connu et conna\^it toujours de nombreuses \'evolutions, \'evolutions que le march\'e de l'\'energie met cependant plus de temps \`a assimiler.
Depuis quelques ann\'ees, le citoyen lambda peut installer des systèmes de production d'\'energie (dans la majorit\'e des cas, des panneaux solaires) \`a son domicile,
et profiter de sa production ou la revendre, en France, \`a EDF. Il pourra bient\^ot stocker cette \'energie encore plus efficacement grâce aux batteries de nouvelle g\'en\'eration (comme la batterie pr\'esent\'ee l'\'et\'e dernier par Tesla).
A l'ère de l'uberisation, on peut penser que les particuliers producteurs d'\'energie auront bient\^ot un accès au march\'e de l'\'energie et pourront y revendre leur production.
Il faut dès \`a pr\'esent r\'efl\'echir \`a cette question, et d\'evelopper les solutions ad\'equates.

\subsection*{Description de la solution}
Le cadre de la blockchain se pr\^ete parfaitement \`a ce problème.
Chaque batterie qui stocke de l'\'energe serait associ\'ee \`a un registre distribu\'e (distributed ledger),
et permettra au citoyen lambda de consulter les sources d'\'energies des particuliers disponibles.
Il pourra alors acheter une partie de cette \'energie, transaction qui sera alors enregistr\'ee sur le registre.
On peut aussi imaginer des transferts d'\'energie entre producteurs d'\'energie :
un producteur d'\'energie \`a Marseille pourra utiliser de l'\'energie \`a Paris pour recharger sa voiture \'electrique.
Son cr\'edit en \'energie sera alors d\'ebit\'e de la quantit\'e r\'ecup\'er\'ee par sa voiture. \\
Dans le cadre de ce projet, nous aimerions fournir une impl\'ementation d'une telle blockchain,
et proposer une API regroupant les fonctions et informations que les batteries devront fournir.

\end{document}
